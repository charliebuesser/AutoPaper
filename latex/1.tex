\section{Einführung}

Die gegenwärtige Diskussion über die Rückkehr der Mitarbeiter ins Büro hat weder an Aktualität noch an Bedeutung verloren. Insbesondere die COVID-19-Pandemie hat die Abläufe in vielen Unternehmen grundlegend verändert und neue Arbeitsmodelle hervorgebracht. Daher ist es von großer Relevanz, die Vor- und Nachteile der Büroarbeit im Vergleich zur Heimarbeit (Homeoffice) genau zu beleuchten. Während einige Unternehmen den direkten Kontakt und die damit verbundene Unternehmenskultur im Büro als unverzichtbar ansehen, haben andere erkannt, dass flexible Arbeitsmodelle effizient sein können. Vor diesem Hintergrund stellen sich wichtige Fragen zu den Auswirkungen solcher Veränderungen auf die Mitarbeiterzufriedenheit, Produktivität und den sozialen Austausch.

\subsection{Hintergrund und Relevanz des Themas}

Die Gestaltung von Arbeitsplätzen hat sich in den letzten Jahren erheblich gewandelt. Laut einer Studie hat die Mehrheit der Unternehmen während der Pandemie Homeoffice-Modelle implementiert, um den Geschäftsbetrieb aufrechtzuerhalten und die Gesundheit der Mitarbeiter zu schützen \cite{file1}. Diese Umstellung stellte Unternehmen nicht nur vor technologische Herausforderungen, sondern erforderte auch ein Umdenken in Bezug auf die Unternehmenskultur und die soziale Interaktion. Dabei hat sich herausgestellt, dass die digitale Zusammenarbeit nicht zwangsläufig zu Einbußen in der Produktivität führen muss. Jedoch kann der Wegfall interpersoneller Kontakte auch negative Effekte auf den Wissensaustausch und die Teambildung mit sich bringen \cite{file2}.

Die Relevanz dieses Themas wird zudem durch die unterschiedlichen Perspektiven von Arbeitnehmern und Arbeitgebern unterstrichen. Während viele Arbeitnehmer die Flexibilität und die Vereinfachung des Arbeitslebens im Homeoffice schätzen, befürchten Arbeitgeber, dass der Verlust physischer Präsenz die Innovationskraft und die Teamdynamik beeinträchtigen könnte. Diese Divergenz in den Sichtweisen verdeutlicht die Notwendigkeit, eine ausgewogene Betrachtung der verschiedenen Modelle zu finden.

\subsection{Zielsetzung und Forschungsfrage}

Ziel dieser Seminararbeit ist es, die Frage zu untersuchen, inwiefern es sinnvoll ist, dass Firmen ihre Angestellten anweisen, wieder im Büro anstatt von zu Hause aus zu arbeiten. Um dies zu erreichen, wird zunächst die Entwicklung der digitalen Arbeitswelt und die Rolle des Homeoffice analysiert. Auf Basis dieser theoretischen Grundlagen werden die Perspektiven sowohl der Arbeitnehmer als auch der Arbeitgeber betrachtet, um schlussendlich die Vorteile und Herausforderungen beider Arbeitsmodelle zu vergleichen.

Die zentrale Forschungsfrage lässt sich wie folgt formulieren: „Inwiefern beeinflussen Homeoffice und Büroarbeit die Produktivität, Mitarbeiterzufriedenheit und Unternehmenskultur?“. Diese Fragestellung wird in den folgenden Kapiteln aus verschiedenen Blickwinkeln beleuchtet, um zu einer differenzierten Antwort zu gelangen.

Zusammenfassend soll die vorliegende Seminararbeit einen fundierten Überblick über die momentanen Herausforderungen und Chancen der Rückkehr ins Büro geben und aufzeigen, wie Unternehmen die Bedürfnisse ihrer Mitarbeiter in einer sich wandelnden Arbeitswelt bestmöglich berücksichtigen können.